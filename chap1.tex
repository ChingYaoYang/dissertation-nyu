\chapter{Improved Stability and Half-Life of Fluorinated Phosphotriesterase
Using Rosetta} 
\label{chap:rosetta}

\begin{refsection}

\section{Introduction}

\subsection{Rosetta and Protein Engineering}
\label{sec:rosetta}

Content here.

\subsection{Phosphotriesterase} 
\label{sec:pte}

Content here.

\subsection{Incorporation of Non-natural Amino Acids} \label{sec:rsi}

Content here.

\subsection{Fluorinated Amino Acids In Proteins} 
\label{sec:faa}

Content here.

\subsection{Scope of Work}

The primary goals of this work were to adapt Rosetta for phosphotriesterase.
Overall, with incorporation of \emph{p}FF into protein, we will be able
evaluate the performance of scoring function. In advance, we would evaluate the
shelf life and thermo-stability of phosphotriesterase.

\section{Methods}

\subsection{General}

All chemicals, reagents, and substrate were purchased from Sigma. T4 DNA ligase
was purchased from Roche. DNA sequence was confirmed by Eurofins MWG Operon.
96-well plates were purchased from Thermo Fisher Scientific (Waltham, MA).

\subsection{Recombinant Gene Construction}

pQE30-S5 was used as described before. The pQE30-104A plasmid
was prepared with forward primers (5’-GATGTGTCGACTGCCGATATCGGTCG-3’, Fisher
Scientific), reverse primers (5’-CGACCGATATCGGCAGTCGACACA-3’, Fisher
Scientific). The PCR parameters were set as follow for 18 cycles: initial
denaturation in \SI{95}{\celsius} for 30 seconds, sequential denaturation in
\SI{95}{\celsius} for 30 seconds, annealing in \SI{55}{\celsius} for 1 minute,
and extension in \SI{68}{\celsius} for 4 minutes. The mixture was then
incubated \SI{37}{\celsius} overnight with DpnI to digest methylated parent DNA
strands, which lack the desired mutation. DNA sequence was further confirmed by
Eurofins MWG Operon.

\subsection{Protein Expression}

Mutant and wild type plasmids were transformed into \latin{E. coli} phenylalanine
auxotrophic strains (AF-IQ cells).[5] Electroporation was done at
\SI{25}{\micro\farad}, \SI{100}{\ohm}, 2.5 kV (Biorad Gene Pulser II). Cells were
plated on agar plates containing \SI{200}{\ug\per\mL} ampicillin, \SI{34}{\ug\per\mL}
chloramphenicol. A Single colony was picked and grown in medium (M9 medium
supplemented with 0.2 wt \% glucose, \SI{35}{\mg\per\L} thiamine, \SI{1}{\milli\Molar}
\ch{MgSO4}, \SI{0.1}{\milli\Molar}\ch{CaCl2}, \SI{200}{\ug\per\mL} ampicillin, and
\SI{34}{\ug\per\mL} chloramphenicol) with \SI{20}{\mg\per\L} of 20 amino acids
at \SI{37}{\celsius}, 300 r.p.m.  Afterwards, \SI{250}{\mL} of M9 medium for
large-scale expression was innoculated 1:50 with an overnight culture.  After
optical density reached 1.0 at 600 nm, media shift was carried out by washing
the cells three times with 0.9\% \SI{4}{\celsius} \ch{NaCl}.  Cells were then
transferred to M9 minimal medium containing either 20 amino acids or 19 amino
acids (-Phe). \emph{p}FF-PTE and \emph{p}FF-104A expression media were
supplemented with and \SI{3}{\milli\Molar} of \emph{p}FF and
\SI{1}{\milli\Molar} isopropyl-$\beta$-D-thiogalactopyranoside (IPTG) to induce
protein expression.  \SI{1}{\milli\Molar} of \ch{CoCl2} was added in each
post-induction medium. After three hours incubation at \SI{37}{\celsius}, 300
r.p.m., the cells were harvested and then resuspended with
\SI{20}{\milli\Molar} Tris-HCl, \SI{500}{\milli\Molar} \ch{NaCl},
\SI{5}{\milli\Molar} imidazole, 10\% glycerol (pH 8.0) and \SI{1}{\micro\Molar}
\ch{CoCl2}. Cell lysate was sonicated on ice for 1.5 minutes and then a
clarification spin was performed (20, 000 g, \SI{4}{\celsius}, 30 min).
Clarified supernatants were loaded into a His Trap column (G.E Healthcare,
Piscataway, NJ) using ÄKTA FPLC purifier (G.E.  Healthcare, Piscataway, NJ).
Protein elution was generated using elution buffer B (\SI{20}{\milli\Molar}
Tris-HCl, \SI{500}{\milli\Molar} sodium chloride, \SI{500}{\milli\Molar}
imidazole (pH 8.0)).  The purified samples were then transferred for buffer
exchange using \SI{12}{\L} \SI{20}{\milli\Molar} phosphate buffer (pH 8.0).
Dialyzed protein was subjected to kinetic assays immediately.

\subsection{PyRosetta Design}
\label{sec:rosetta-method}

Content here.

\subsection{Thermo-stability and Secondary Structure of Phosphotriesterase}
\label{sec:thermo}

Content here.

\subsubsection{Nano-DSC}
\label{sec:dsc}

DSC (Nano-DSC, TA instrument, USA) was preformed by using \SI{600}{\micro\L}
(\SI{0.1}{\mg\per\mL}) of protein right after dialysis. Measurements were
conducted at a scan rate of \SI{1}{\celsius\per\minute}. Signals was blanked with
buffer under the same condition.  The observed diagram was then analyzed by
using NanoAnalyze software.

\subsubsection{Circular Dichroism}
\label{sec:cd}

CD spectra were recorded on a JASCO J-815 Spectropolarimeter (Easton, MD) using
Spectra Manager software. Temperature was controlled using a Fisher Isotemp
Model 3016S water bath. Proteins concentrations were \SI{10}{\micro\Molar} in
\SI{20}{\milli\Molar} phosphate buffer (pH 8.0). \SI{20}{\milli\Molar}
phosphate buffer was used for blanking signals. To calculate ellipticities, the
following formula was used(Eq.~\ref{eqn:CD}): 
\begin{equation}
    θmrw = MRW(θobs) / (10 * c * l) 
    \label{eqn:CD}
\end{equation}
where \emph{MRW} is the mean residue weight of the specific phosphotriesterase,
θobs is the observed ellipticities (mdeg), \emph{l} is the path length (cm),
\emph{c} is the concentration in \SI{}{\micro\Molar}. Spectra was recorded from
\SIrange{190}{250}{\nm} with a scan speed of \SI{1}{\nano\meter\per\minute}.

\subsection{Enzyme Kinetics}
\label{sec:kinetics}

The protein was diluted to a final concentration of \SI{30}{\nano\Molar} in
\SI{20}{\milli\Molar} sodium phosphate (pH 8.0) by using the extinction
coefficient \SI{29280}{\per\Molar\per\cm}. Reactions were monitored
spectrophotometrically (Synergy H1, BioTek, Winooski VT) at \SI{405}{\nm} for
paraoxon (coefficient = \SI{17000}{\per\Molar\per\cm}).  Reactions for paraoxon
(\SIrange{13}{104}{\micro\Molar}) was done in 0.4\% methanol.
K\textsubscript{M} and k\textsubscript{cat} values were determined by a
Lineweaver-Burk plot.\cite{Baker2011b} The equation used is shown below
(Eq.~\ref{eqn:MM}): 
\begin{equation} 
    \frac{1}{v} =
    \frac{K\textsubscript{M}}{V\textsubscript{max}}\times\frac{1}{S} +
    \frac{1}{V\textsubscript{max}} 
    \label{eqn:MM}
\end{equation}
where S represents substrate concentration; K\textsubscript{M} represents the
substrate concentration at which the reaction rate is half of
V\textsubscript{max}. The data reported is the average of three trials and the
error represents the standard deviation of those trials.

\subsection{MALDI-TOF Mass Spectrometry}

To determine level of \emph{p}FF incorporation, \SI{20}{\micro\liter} of
purified PTE \emph{p}FF-PTE, F104A, or \emph{p}FF-104A was incubated with
\SI{12.5}{\ng\per\uL} of trypsin solution (in \SI{50}{\milli\Molar} of ammonium
bicarbonate) at \SI{37}{\celsius} overnight. \SI{2}{\uL} of 10\%
trifluoroacetic acid (TFA) was used to quench each reaction. Reaction was then
purified with a C\textsubscript{18} packed zip-tip (Millipore, Billerica, MA).
Tips were wetted in 50\% acetonitirile (ACN), equilibrated in 0.1\% TFA, and
eluted with 0.1\% TFA in 75\% ACN. Matrix was dissolved in \SI{10}{\mg\per\mL}
$\alpha$-cyano-4-hydrocinnamic acid (CCA) in 50\% ACN, 0.05\% TFA. Theoretical
trypsin digest were calculated from Peptide Mass
(www.expasy.org/tools/peptide-mass.html). Samples were added to the matrix at a
1:1 ratio and spotted on MALDI plate. Five standards were spotted separately
for calibration: angiotensin I (MW = \SI{1295.69}{\g\per\mole}), neurotensin
(MW = \SI{1671.92}{\g\per\mole}), ACTH (1-17) (MW = \SI{2092.09}{\g\per\mole}),
ACTH (18-39) (MW = \SI{2464.20}{\g\per\mole}), and ACTH (7-38) (MW =
\SI{3656.93}{\g\per\mole}).  Compass 1.4 for flex software was then used to
analyze the MALDI spectra (www.bruker.com/).

\section{Results and Discussion}

\subsection{Biosynthesis of Phosphotriesterase}

Content here.

\subsection{Thermo-stability And Secondary Structure}

Content here.

\subsection{Enzymatic Kinetics of Phosphotriesterase}

Content here.

\subsection{Protein Design}

Content here.

\printbibliography[heading=subbibliography]

\end{refsection}
